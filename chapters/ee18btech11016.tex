An amplifier has a dc gain of $10^5$ and poles at $10^5$ Hz , $3.16 \times 10^5$ Hz and $10^6$ Hz . Find the value of $\beta$,and the corresponding closed-loop gain , for which a phase margin of $45\degree$ is obtained. 

\begin{enumerate}[label=\arabic*.,ref=\theenumi]
\numberwithin{equation}{enumi}

\item Find the transfer function of the three pole OPAMP.
\\
\solution 
For a 3-pole amplifier open loop transfer function is 
\begin{align}
G(s) = \frac{G_0}{\brak{1+\frac{s}{P_{1}}}\brak{1+\frac{s}{P_{2}}}\brak{1+\frac{s}{P_{3}}}}
\end{align}

where the Gain and Poles are listed in Table \ref{table:ee18btech11016_Table_1}.
%
\begin{table}[!ht]
\centering
\input{./tables/ee18btech11016_1.tex}
\caption{}
\label{table:ee18btech11016_Table_1}
\end{table}

Poles are at $f_{1} =10^5$ and $f_{2} = 3.16\times10^{5}$ and $f_{3} = 10^6$ 
\begin{multline}
G\brak{f} = \frac{G_{0}}{\brak{1+\j\frac{f}{f_{1}}}\brak{1+\j\frac{f}{f_{2}}}\brak{1+\j\frac{f}{f_{3}}}}\\
= \frac{10^{5}}{\brak{1+\j\frac{f}{10^5}}\brak{1+\j\frac{f}{3.16\times10^{5}}}\brak{1+\j\frac{f}{10^6}}}
\label{eq:ee18btech11016_1}
\end{multline}
\item Find the loop gain expression ($G(s)H$) (H is constant in this question).
\\
\solution 
\begin{align}
GH = \frac{10^{5}}{\brak{1+\j\frac{f}{10^5}}\brak{1+\j\frac{f}{3.16\times10^{5}}}\brak{1+\j\frac{f}{10^6}}} . H
\label{eq:ee18btech11016_2}
\end{align}

\item Find the PM and the crossover frequency.
\\
\solution  
The phase margin = $180\degree$ - $\phi(f_{c})$ where $f_{c}$ is the frequency where $\abs{G(f)H}=1$.It is required that the phase margin is $45\degree$ , so that :

\begin{align}
45\degree &=  180\degree - \phi(f_{c}) \implies \phi(f_{c})=-135\degree.
\end{align}
From \eqref{eq:ee18btech11016_2} 
\begin{align}
-135\degree = -\tan^{-1}\brak{\frac{f_{c}}{10^5}}-\tan^{-1}\brak{\frac{f_{c}}{3.16\times10^5}}-\tan^{-1}\brak{\frac{f_{c}}{10^6}}
\end{align}
After solving the above equation , we get $f_{c}$ = 315KHz.
\\

\item Verify your result using a Bode plot.
\\
\solution  The following code  id used to verify the value of $f_{c}$ Fig. \ref{fig:ee18btech11016_1}

\begin{lstlisting}
codes/ee18btech11016/ee18btech11016_1.py
\end{lstlisting}
%
\begin{figure}[!h]
\centering
\includegraphics[width=\columnwidth]{./figs/ee18btech11016/ee18btech11016_resultbode.eps}
\caption{}
\label{fig:ee18btech11016_1}
\end{figure}

\item Find the value of H.\\
\solution From \eqref{eq:ee18btech11016_2},The magnitude of the loop gain at this frequency $f_{c}$ is given by $\abs{G(f_{c})H}$ :
 
\begin{align}
    H\brak{\frac{10^5}{\sqrt{1+(\frac{315\times10^3}{10^5})^2}\sqrt{1+(\frac{315\times10^3}{3.16\times10^6})^2}\sqrt{1+(\frac{315\times10^3}{10^6})^2}}}
\end{align}

which is equal to $H\times(20.04\times10^3)$.
So,
\begin{align}
\abs{G(f_{c})H} = H\times(20.04\times10^3)
\end{align}
Setting $\abs{G(f_{c})H}$ = 1 , Solving for $\beta$ (here $\beta$ is equal to H) yields 
\begin{align}
\beta &= 48.9\times10^{-6}
\end{align}
Or
\begin{align}
H &= 48.9\times10^{-6}
\end{align}

\item Find the corresponding closed-loop gain for which a phase margin of $45\degree$ is obtained.\\
\solution The closed loop dc gain is given as
\begin{align}
A_{f} = \frac{G_{0}}{1+HG_{0}}=\frac{10^5}{1+48.9\times10^{-6}(10^5)}
\end{align} 
\begin{align}
A_{f} = 17\times10^3
\end{align}


\item Realise the above system with $PM = 45 \degree$ using a feedback circuit.\\
\solution
\begin{figure}[ht!]
	\begin{center}
		\resizebox{\columnwidth/1}{!}{\input{./figs/ee18btech11016/ee18btech11016_figa.tex}}
	\end{center}
	\caption{}
	\label{fig:ee18btech11016_figa}
\end{figure}

The transfer function of OPAMP is
\begin{align}
    G(s) = \frac{10^{5}}{\brak{1+\frac{s}{2\pi \times 10^5}}\brak{1+\frac{s}{2\pi \times3.16 \times 10^{5}}}\brak{1+\frac{s}{2\pi \times 10^{6}}}}
\end{align}
%
\item For the feedback gain H\\
\solution\\
Choose a resistance network such that
\begin{align}
    H = \frac{V_{f}}{V_{o}} = \frac{R_{f_{1}}}{R_{f_{1}}+R_{f_{2}}} \approx 48.9\times10^{-6}
\end{align}
\begin{figure}[ht!]
	\begin{center}
		\resizebox{\columnwidth/2}{!}{\input{./figs/ee18btech11016/ee18btech11016_figb.tex}}
	\end{center}
	\caption{}
	\label{fig:ee18btech11016_figb}
\end{figure}

Choose $R_{f_{1}}$ and $R_{f_{2}}$ as
\begin{align}
    R_{f_{1}} = 100\ohm\\
    R_{f_{2}} = 2.045M\ohm
\end{align}
\begin{align}
H = \frac{R_{f_{1}}}{R_{f_{1}}+R_{f_{2}}} = \frac{100}{100+2.045\times10^6} \approx 48.9\times10^{-6}
\end{align}
\item Feedback Circuit for $PM = 45\degree$
\\
\solution
\begin{figure}[ht!]
	\begin{center}
		\resizebox{\columnwidth}{!}{\input{./figs/ee18btech11016/ee18btech11016_figc.tex}}
	\end{center}
	\caption{}
	\label{fig:ee18btech11016_figc}
\end{figure}

\item Simulate the circuit in ngspice.\\
\solution The following code provides instructions about the simulation.
\begin{lstlisting}
codes/ee18btech11016/spice/README.md
\end{lstlisting}


The following netlist simulates the unity feedback system. 
\\

\begin{lstlisting}
codes/ee18btech11016/spice/ee18btech1016_sim.net
\end{lstlisting}
 The step response in spice is plotted using the following code in Fig. \ref{fig:ee18btech11016_spice}
 \begin{lstlisting}
codes/ee18btech11016/spice/ee18btech11016_simulation.py
\end{lstlisting}
\begin{figure}[!h]
\centering
\includegraphics[width=\columnwidth]{./figs/ee18btech11016/ee18btech11016_spice.eps}
\caption{}
\label{fig:ee18btech11016_spice}
\end{figure}

\item Overview of implementation.\\
\solution Fig. \ref{fig:ee18btech110016_circuit_1} shows how the circuit is actually implemented in spice using the parameters in Table \ref{table:ee18btech11016_Table_2}  

\begin{figure}[!ht]
	\begin{center}
				\resizebox{\columnwidth}{!}{\input{./figs/ee18btech11016/circuit_1.tex}}
	\end{center}
\caption{Circuit resembling G(s)}
\label{fig:ee18btech110016_circuit_1}
\end{figure}

\begin{table}[!ht]
\centering
\input{./tables/ee18btech11016_table2.tex}
\caption{}
\label{table:ee18btech11016_Table_2}
\end{table}

\end{enumerate}
